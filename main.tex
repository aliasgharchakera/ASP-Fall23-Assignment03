\documentclass[answers]{exam}

\usepackage{amsmath}
\usepackage{amssymb}
\usepackage{geometry}
\usepackage{venndiagram}
\usepackage{graphics}
\usepackage{graphicx}
\usepackage{tikz}

% Header and footer.
\pagestyle{headandfoot}
\runningheadrule
\runningfootrule
\runningheader{Applied Stochastic Processes}{Assignment 3}{Fall 2023 }
\runningfooter{}{Page \thepage\ of \numpages}{}
\firstpageheader{}{}{}

\boxedpoints
\printanswers

\newcommand{\uvec}[1]{\boldsymbol{\hat{\textbf{#1}}}}
\newcommand\union\cup
\newcommand\inter\cap
\newcommand\ul\underline
\newcommand\ol\overline

\title{Assignment 3\\ Applied Stochastic Processes\\ Habib University -- Fall 2023}
\author{Ali Asghar Yousuf - ay06993 \\ Muhammad Murtaza - mm06369 }  % replace with your ID, e.g. oy02945
\date{\today}

\begin{document}
\maketitle

\begin{questions}
    \question Dave fails quizzes with probability $\dfrac{1}{4}$, independent of other quizzes.
    \begin{parts}
        \part What is the probability that Dave fails exactly two of the next six quizzes?
        \begin{solution}
            \begin{align*}
                P(\text{Dave fails exactly two of the next six quizzes}) & = \binom{6}{2} \left(\frac{1}{4}\right)^2 \left(\frac{3}{4}\right)^4 \\
                                                                         & = 15 \times \frac{1}{16} \times \frac{81}{256}                       \\
                                                                         & = \frac{1215}{4096}                                                  \\
                                                                         & = 0.296875
            \end{align*}
        \end{solution}
        \part What is the expected number of quizzes that Dave will pass before he has failed
        three times?
        \begin{solution}

            No. of times he failed $= 3$ \\ Total no. of quizzes taken to fail 3 times $=
                n$ \\
            \begin{align*}
                n * \frac{1}{4} = 3 \\
                n = 12
            \end{align*}
            Dave takes 12 quizzes to fail 3 times. Therefore, he passes 9 quizzes. \\
        \end{solution}
        \part What is the probability that the second and third time Dave fails a quiz will
        occur when he takes his eighth and ninth quizzes, respectively?
        \begin{solution}

            1st Fail $\rightarrow 1 - 7$ quizzes \\
            2nd Fail $\rightarrow 8$th quiz \\
            3rd Fail $\rightarrow 9$th quiz \\

            \begin{align*}
                P(X) & =P(\text{1 fail in 7 tests}) \cdot P(\text{2nd fail in 8th test}) \cdot P(\text{3rd fail in 9th test}) \\
                     & =\binom{7}{1}\left(\frac{1}{4}\right)^1 \left(\frac{3}{4}\right)^6 \cdot \frac{1}{4} \cdot \frac{1}{4} \\
                     & =\frac{7 \cdot 3^6}{4^9} = \frac{5103}{262144}                                                         \\
                     & =0.0194568634
            \end{align*}
        \end{solution}
        \part What is the probability that Dave fails two quizzes in a row before he passes two
        quizzes in a row?
        \begin{solution}

            $F =$ Fail, $P =$ Pass \\
            \begin{align*}
                P(X) & = P(\text{Dave fails two quizzes in a row before he passes two quizzes in a row})            \\                                                                                                                       
                     & = P(FF \cup PFF \cup FPFF \cup PFPFF \cup FPFPFF \cup \dots)                                 \\                                                                                                                      
                     & = \dfrac{[P(F)]^2}{1 - P(F) \cdot P(P)} + \dfrac{P(P) \cdot [P(F)]^2}{1 - P(F) \cdot P(P)}   \\
                     & = \dfrac{\left(\frac{1}{4}\right)^2}{1 - \frac{1}{4} \cdot \frac{3}{4}} + \dfrac{\frac{3}{4} \cdot \left(\frac{1}{4}\right)^2}{1 - \frac{1}{4} \cdot \frac{3}{4}} \\
                     & = \dfrac{7}{52}
            \end{align*}

        \end{solution}
    \end{parts}

    
\end{questions}

\end{document}
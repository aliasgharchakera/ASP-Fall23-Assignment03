\documentclass[answers]{exam}

\usepackage{amsmath}
\usepackage{amssymb}
\usepackage{geometry}
\usepackage{venndiagram}
\usepackage{graphics}
\usepackage{graphicx}
\usepackage{tikz}

% Header and footer.
\pagestyle{headandfoot}
\runningheadrule
\runningfootrule
\runningheader{Applied Stochastic Processes}{Assignment 3}{Fall 2023 }
\runningfooter{}{Page \thepage\ of \numpages}{}
\firstpageheader{}{}{}

\boxedpoints
\printanswers

\newcommand{\uvec}[1]{\boldsymbol{\hat{\textbf{#1}}}}
\newcommand\union\cup
\newcommand\inter\cap
\newcommand\ul\underline
\newcommand\ol\overline

\title{Assignment 3\\ Applied Stochastic Processes\\ Habib University -- Fall 2023}
\author{Ali Asghar Yousuf - ay06993 \\ Muhammad Murtaza - mm06369 }  % replace with your ID, e.g. oy02945
\date{\today}

\begin{document}
\maketitle
\section{Bertsekas and Tsitsiklis, Section 7.1}
\begin{questions}
    \question \textbf{Problem 2}

    Dave fails quizzes with probability $\dfrac{1}{4}$, independent of other
    quizzes.
    \begin{parts}
        \part What is the probability that Dave fails exactly two of the next six quizzes?
        \begin{solution}
            \begin{align*}
                P(\text{Dave fails exactly two of the next six quizzes}) & = \binom{6}{2} \left(\frac{1}{4}\right)^2 \left(\frac{3}{4}\right)^4 \\
                                                                         & = 15 \times \frac{1}{16} \times \frac{81}{256}                       \\
                                                                         & = \frac{1215}{4096}                                                  \\
                                                                         & = 0.296875
            \end{align*}
        \end{solution}
        \part What is the expected number of quizzes that Dave will pass before he has failed
        three times?
        \begin{solution}

            No. of times he failed $= 3$ \\ Total no. of quizzes taken to fail 3 times $=
                n$ \\
            \begin{align*}
                n * \frac{1}{4} = 3 \\
                n = 12
            \end{align*}
            Dave takes 12 quizzes to fail 3 times. Therefore, he passes 9 quizzes. \\
        \end{solution}
        \part What is the probability that the second and third time Dave fails a quiz will
        occur when he takes his eighth and ninth quizzes, respectively?
        \begin{solution}

            1st Fail $\rightarrow 1 - 7$ quizzes \\
            2nd Fail $\rightarrow 8$th quiz \\
            3rd Fail $\rightarrow 9$th quiz \\

            \begin{align*}
                P(X) & =P(\text{1 fail in 7 tests}) \cdot P(\text{2nd fail in 8th test}) \cdot P(\text{3rd fail in 9th test}) \\
                     & =\binom{7}{1}\left(\frac{1}{4}\right)^1 \left(\frac{3}{4}\right)^6 \cdot \frac{1}{4} \cdot \frac{1}{4} \\
                     & =\frac{7 \cdot 3^6}{4^9} = \frac{5103}{262144}                                                         \\
                     & =0.0194568634
            \end{align*}
        \end{solution}
        \part What is the probability that Dave fails two quizzes in a row before he passes two
        quizzes in a row?
        \begin{solution}

            $F =$ Fail, $P =$ Pass \\
            \begin{align*}
                P(X) & = P(\text{Dave fails two quizzes in a row before he passes two quizzes in a row})                                                                                 \\
                     & = P(FF \cup PFF \cup FPFF \cup PFPFF \cup FPFPFF \cup \dots)                                                                                                      \\
                     & = \dfrac{[P(F)]^2}{1 - P(F) \cdot P(P)} + \dfrac{P(P) \cdot [P(F)]^2}{1 - P(F) \cdot P(P)}                                                                        \\
                     & = \dfrac{\left(\frac{1}{4}\right)^2}{1 - \frac{1}{4} \cdot \frac{3}{4}} + \dfrac{\frac{3}{4} \cdot \left(\frac{1}{4}\right)^2}{1 - \frac{1}{4} \cdot \frac{3}{4}} \\
                     & = \dfrac{7}{52}
            \end{align*}

        \end{solution}
    \end{parts}

    \question \textbf{Problem 3}

    A computer system carries out tasks submitted by two users. Time is divided
    into slots. A slot can be idle, with probability $P_I = \frac{1}{6}$, and busy
    with probability $P_B = \frac{5}{6}$. During a busy slot, there is probability
    $P_{1 | B} = \frac{2}{5}$ (respectively, $P_{2 | B} = \frac{3}{5}$) that a task
    from user 1 (respectively, 2) is executed. We assume that events related to
    different slots are independent. \\ $T_1 = $ Task from user 1.

    \begin{parts}
        \part Find the probability that a task from user 1 is executed for the first time during
        the 4th slot.
        \begin{solution}
            If a task from user 1 is executed for the first time during the 4th slot, then the task from user 1 is not executed in the first 3 slots
            (they are busy and not accepting from user 1) and is executed in the 4th slot (4th slot maybe idle and execute or busy and execute). \\
            \begin{align*}
                 & P(T_1 \text{ is executed for the first time during the 4th slot})                                                                                  \\
                 & = P(T_1 \text{ is not executed in the first 3 slots}) \cdot P(T_1 \text{ is executed in the 4th slot})                                             \\
                 & = \left(\frac{5}{6} \times \frac{3}{5}\right)^3 \cdot \left[\left(\frac{1}{6} \times 1\right) + \left(\frac{5}{6} \times \frac{2}{5}\right)\right] \\
                 & = \left(\frac{1}{2}\right)^3 \cdot \left[\frac{1}{6} + \frac{1}{3}\right]                                                                          \\
                 & = \frac{1}{8} \cdot \frac{1}{2}                                                                                                                    \\
                 & = \frac{1}{16}                                                                                                                                     \\
            \end{align*}
        \end{solution}

        \part Given that exactly 5 out of the first 10 slots were idle, find the probability that
        the 6th idle slot is slot 12.
        \begin{solution}
            Since exactly 5 out of the first 10 slots were idle, therefore, the 11th slot is busy. \\
            And since the slots are independent, \\
            \begin{align*}
                 & P(\text{6th idle slot is slot 12})                              \\
                 & = P(\text{11th slot is busy}) \cdot P(\text{12th slot is idle}) \\
                 & = \frac{5}{6} \times \frac{1}{6}                                \\
                 & = \frac{5}{36}                                                  \\
                 & = 0.138889
            \end{align*}
        \end{solution}

        \part Find the expected number of slots up to and including the 5th task from user 1.
        \begin{solution}
            Probability of a task from user 1.
            \begin{align*}
                P(T_1) & = P_I \cdot P_{1|I} + P_B \cdot P_{1|B}               \\
                       & = \frac{1}{6} \cdot 1 + \frac{5}{6} \cdot \frac{2}{5} \\
                       & = \frac{1}{6} + \frac{1}{3}                           \\
                       & = \frac{1}{2}
            \end{align*}
            Probability of 5th task from user 1 (first 4 slots are busy and not accepting from user 1).
            \begin{align*}
                P(\text{5th task from user 1}) & = \left(\frac{1}{2}\right)^5 \\
                                               & = \frac{1}{32}
            \end{align*}
            Expected number of slots up to and including the 5th task from user 1 is the reciprocal of the probability of 5th task from user 1 $= 32$
        \end{solution}
        \part Find the expected number of busy slots up to and including the 5th task from
        user 1.
        \begin{solution}
            In busy slots, there is probability $P_{1 | B} = \frac{2}{5}$ that a task from user 1 is executed. \\
            Probability of 5th task from user 1 $ = \left(\frac{5}{6}\right)^5 \cdot \frac{2}{5} = 0.16075$ \\
            Expected number of busy slots up to and including the 5th task from user 1 is the reciprocal of the probability of 5th task from user 1 $= \frac{1}{0.16075} = 6.219$
        \end{solution}

        \part Find the PMF, mean, and variance of the number of tasks from user 2 until the
        time of the 5th task from user 1.
        \begin{solution}
            \begin{align*}
                \binom{k+r-1}{k} p^{k} (1-p)^{r}                                                              \\
                p               & = \left( \frac{1}{6} + \frac{5}{6} \times \frac{3}{5} \right) = \frac{2}{3} \\
                \binom{k+r-1}{k} \left(\frac{2}{3}\right)^{k} \left(\frac{1}{3}\right)^{r}                    \\
                \text{Mean}     & = \frac{pr}{1-p} = 2r                                                       \\
                \text{Variance} & = \frac{pr}{(1-p)^{2}} = 6r                                                 \\
            \end{align*}

            The expression $\binom{k+r-1}{k} p^{k} (1-p)^{r}$ represents the probability of
            having $k$ successes and $r$ failures in a sequence of trials, where each trial
            has a success probability of $p$ and a failure probability of $1-p$ (Binomial
            distribution).

            In this case, $p$ is calculated as $\left( \frac{1}{6} + \frac{5}{6} \times
                \frac{3}{5} \right) = \frac{2}{3}$.

            The mean and variance of this distribution are given by $\text{Mean} =
                \frac{pr}{1-p} = 2r$ and $\text{Variance} = \frac{pr}{(1-p)^{2}} = 6r$,
            respectively.
        \end{solution}
    \end{parts}
\end{questions}

\section{Leon-Garcia, Section 11}
\begin{questions}
    \question \textbf{11.9}

    Let $X_n$ be an iid integer-valued random process. Show that $X_n$ is a Markov process and
    give its one-step transition probability matrix.
    \begin{solution}
        
    \end{solution}
\end{questions}

\end{document}